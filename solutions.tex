\documentclass{article}
\usepackage{amsfonts} 
\usepackage{amsmath}
\usepackage{amssymb}
\usepackage{dcolumn}

\newcolumntype{2}{D{.}{}{4.0}}
\title{Solutions: Galois Theory by Tom Leinster}
\author{Hassaan Naeem}
\date{\today}
\begin{document}
\maketitle
\section*{Chapter 1. Overview of Galois Theory}
\subsection*{Exercise 1.1.3} 
Both proofs of ‘if’ contain little gaps: ‘It follows by
induction’ in the first proof, and ‘it’s easy to see’ in the second. Fill
them.\

\paragraph{Solution:} We show both both parts (i) and (ii) seperately

\begin{enumerate}
    \item[(i)] Follows from induction that for any polynomial $p$ over $\mathbb{R}$, $\overline{ p(w)} = p(\overline w)$:\\
        \linebreak 
        Let $p(w) = c_0 + c_1w^1 + c_2w^2 + ... + c_nw^n$ where $w^n \in \mathbb{C}$ and $c_n \in \mathbb{C}$. \\
            \begin{equation*}
                \begin{aligned}
                    \overline{p(w)} & = \overline{c_0 +c_1w^1 + c_2w^2 + ... + c_nw^n}\\
                    & = \overline{c_0} + \overline{c_1w^1} + \overline{c_2w^2} + ... + \overline{c_nw^n}\\
                    & = c_0 + c_1\overline{w^1} + c_2\overline{w^2} + ... + c_n\overline{w^n}\\
                    & = p(\overline w)
                \end{aligned}
            \end{equation*}
    \item[(ii)] Checking that r is the zero polynomial:\\
        \linebreak
        \textbf{Lemma}. If $r(x) = a_0 + a_1x^1+ ... + a_nx^n $ and $r(x) = 0,   \forall x \neq 0 $, then $a_0 = 0$. Since $\mathbb{Q}$ is a field, and must contain a zero.\\
        
        By this lemma we have that $\forall x$, $ r(x) = 0$ and therefore\\
        $r(x) = 0 = x(a_1+ a_2x + ... + a_nx^{n-1}) = a_1+ a_2x + ... + a_nx^{n-1}$ and $\forall x \neq 0$ $a_1 = 0$.
        Hence, we repeat the Lemma and show that all $a_1, ..., a_n = 0$. Therefore $r(x)$ is the zero polynomial.
\end{enumerate}

\subsection*{Exercise 1.1.6} 
Let $z \in \mathbb{Q}$. Show that $z$ is not conjugate to $z'$ for any complex number $z' \neq z$.

\paragraph{Solution:}

\subsection*{Exercise 1.1.10}
Suppose that $(z_1, ..., z_k)$ and $(z_1', ..., z_k')$ are conjugate. Show that $z_i$ and $z_i'$ are conjugate, for each $i \in \{1, ..., k\}$

\paragraph{Solution:} By \textbf{Definition 1.1.9} have that:
    \begin{equation*}
        \begin{aligned}
            p(z_1, ..., z_k) & = 0 \Longleftrightarrow p(z_1', ..., z_k')=0
        \end{aligned}
    \end{equation*}
\noindent
When $k=1$ we have that:
\\\\
\centerline{$p(z_1)=0 \Longleftrightarrow p(z_1') = 0 \implies$  $z_1$ and $z_1'$ are conjugate}
\\\\
Similarly, for any $k$:
\\\\
\centerline{$p(z_i)=0 \Longleftrightarrow p(z_i') = 0 \implies$  $z_i$ and $z_i'$ are conjugate}

\subsection*{Exercise 1.2.2}
Show that $Gal(f)$ is a subgroup of $S_k$.


\section*{Chapter 2. Group actions, rings and fields}

\subsection*{Exercise 2.1.3}
Check that $\bar g$ is a bijection for each $g \in G$. Also check that $\Sigma$ is a homomorphism.

\paragraph{Solution:} We show injectivity (i), surjectivity (ii) and homomorphism (iii) :

    \begin{enumerate}
        \item[(i)] Injectivity:\\
            \linebreak
            Let $x,y \in X$ and  $\bar g$ be our bijection\\
            If we have $\bar g(x) = \bar g(y)$
                \begin{equation*}
                    \begin{aligned}
                        \leadsto gx &= gy \leadsto g^{-1}(gx) = g^{-1}(gy) \leadsto (g^{-1}g)x = (g^{-1}g)y \leadsto ex = ey \leadsto x=y\quad \square\\
                    \end{aligned}
                \end{equation*}
        \item[(ii)] Surjectivity:\\
            \linebreak
            We know that $f: X \rightarrow Y$ is surjective iff $\forall y \in Y\ \exists x \in X: f(x)=y$ \\
            We let $x \in X$ and $e$ be the identity in $G$ then, \\
            $x = ex = (gg^{-1})x = g(g^{-1}x) = gy = \bar g(y)$ where $ y = g^{-1}x \in X \quad \square$ \\ 
            \\Therefore $\bar g$ is both injective and surjective, hence bijective. \\
        \item[(i)] $\Sigma$ is Homomorphism:\\
            \linebreak
            We have the map:
            \begin{equation*}
                \begin{aligned}
                     \Sigma:\ &G \rightarrow Sym(X)\\
                              & g \mapsto \bar g
                \end{aligned}
            \end{equation*}
            We know that $\bar g$ is well defined\\
            Then we take $g, h \in G, x \in X$, then by \textbf{Definition 2.1.1}.\\
            \begin{equation*}
                \begin{aligned}
                    \Sigma(gh)(x)& = gh(x) = g(hx)\\
                                 & = \Sigma(g)(\Sigma(h)(x)) = \Sigma(g) \circ \Sigma(h) \ (x) \quad \square
                \end{aligned}
            \end{equation*}
        \end{enumerate}

\subsection*{Exercise 2.1.10}
Example \textbf{2.1.9(iii)} shows that the action of the isometry
cube $G$ of the cube on the set $X$ of long diagonals is not faithful.
By \textbf{Lemma 2.1.8}, there must be some non-identity isometry of the
cube that fixes all four long diagonals. In fact, there is exactly one.
What is it?

\paragraph{Solution:} We show both both parts (i) and (ii) seperately


\subsection*{Exercise 2.2.6}
Prove that the only subring of a ring $R$ that is also an
ideal is $R$ itself.

\paragraph{Solution:} We know that $I$ is an ideal of $R$ if:\\
    \begin{equation*}
        \begin{aligned}
            &(I, +) \le (R, +)\quad  \text{[I is additive subgroup of R]}\\
            & \&\ \forall r \in R, x \in I:\\
            & (1)\ r \cdot x \in I\\
            & (2)\ x \cdot r \in I
        \end{aligned}
    \end{equation*}

    We know that a subring $S$ of $R$ is a subset $S \subseteq R$ containing $0$ and $1$.\\
    \\Therefore if we take $S$ to be an ideal aswell then:
    \begin{equation*}
        \begin{aligned}
            &(S, +) \le (R, +)\\
            & \&\ \forall r \in R, s \in S:\\
            & (1)\ r \cdot s \in S\\
            & (2)\ s \cdot r \in S
        \end{aligned}
    \end{equation*}
    But we know that $1 \in S$. Therefore, $\forall r \in R,$ $(1)\ 1 \cdot r = r \in S$ and $(2)\ r \cdot 1 = r \in S$\\
    Therefore $\forall r \in R, r \in S \implies S = R \quad \square$

\subsection*{Exercise 2.2.8}
The trivial ring or zero ring is the one-element set
with its only possible ring structure. Show that the only ring in which
$0 = 1$ is the trivial ring.

\paragraph{Solution:} Let $(R, +, \cdot)$ be our commutative, unital ring.\\
If $1=0$ in $R$, then $\forall r \in R$ we have $r= 1r = 0r = 0\quad \square$

\subsection*{Exercise 2.2.8}
Fill in the details of Example 2.2.13.

\paragraph{Solution:} We suppose that $I \subseteq \mathbb{Z}$ is an ideal and we take $n \in I$ to 
be the least positive integer in $I$. We have obviously that $\langle n \rangle \subseteq I$. Then we assume that
that $m \in I$, by the division algorithm we know that:

    \begin{equation*}
        \begin{aligned}
            m &= qn +r  \quad &(0 \leq r < n)\\
            r &= m - qn \quad &\in I
        \end{aligned}
    \end{equation*}
Therefore $r=0$ $\leadsto m = qn$. Therefore $m \in \langle n \rangle$ and we have that $I \subseteq \langle n \rangle$.
Hence we have equality, $I = \langle n \rangle \quad \square$


\subsection*{Exercise 2.2.15}
Let $r$ and $s$ be elements of an integral domain.\\
Show that $r | s |r \Longleftrightarrow \langle r \rangle = \langle s \rangle \Longleftrightarrow s = ur$ for some unit $u$.
\paragraph{Solution:}
If we have that $r|s|r$ then $\exists\ a \in R: s = ar$ and $ \exists\ b \in R: r = bs$ then:
    \begin{equation*}
        \begin{aligned}
            \frac{s}{a} &= bs\\
            b &= \frac{1}{a} \leadsto ab=1 \leadsto b = a^{-1}
        \end{aligned}
    \end{equation*}
Then we have that $s=ar$, and we have just shown that a is a unit, hence $s=ur$. Therefore $r|s|r \implies s = ur$
\\\\
If we have $\langle r \rangle = \langle s \rangle$, then $r=s$. Hence, 
\begin{equation*}
    \begin{aligned}
        r = 1s \quad &\&\quad s = 1r\\
        r = as \quad &\&\quad s = ar\quad (\text{where}\ a=1)\\
        \implies s|r  \ &\&\quad r|s
    \end{aligned}
\end{equation*}
Therefore $\langle r \rangle = \langle s \rangle \implies r|s|r$
\\\\
If we have that $s=ur$ for some unit $u$, then also we have that
\begin{equation*}
    \begin{aligned}
        &u^{-1}s = u^{-1}ur \leadsto r = u^{-1}s\\
        &\text{Therefore}\ s \in \langle r \rangle\ \&\ r \in \langle s \rangle, \langle s \rangle \subseteq \langle r \rangle\ \&\ \langle r \rangle \subseteq \langle s \rangle \\
        &\implies  \langle r \rangle = \langle s \rangle
    \end{aligned}
\end{equation*}
Therefore $s=ur \implies \langle r \rangle = \langle s \rangle$

\subsection*{Exercise 2.3.1}
Write down all the examples of fields that you know.
\\\\
\textbf{Solution:}
$\mathbb{C}, \mathbb{R}, \mathbb{Q}$

\subsection*{Exercise 2.3.5}
Let $\phi : K \rightarrow L$ be a homomorphism of fields and let
$0 \neq a \in K$. Prove that $\phi(a^{-1}) = \phi(a)^{-1}$.
Why is $\phi(a)^{-1}$ defined?
\\\\
\textbf{Solution:}
Since $K$ is a field, and the fact that $0 \neq a \in K$, we have that $a$ is a unit, $aa^{-1}=1$, and $a^{-1} \in K$.
By \textbf{Lemma 2.3.3}, we have that $\phi: K \rightarrow L$ is injective. Hence, $\phi(a)\phi(a^{-1}) = \phi(a \circ a^{-1}) = \phi(1) = 1$, and
$\phi(a^{-1})\phi(a) = \phi(a^{-1} \circ a) = \phi(1) = 1$. Therefore we have that $\phi(a^{-1})$ is both a left and right inverse
of $a$ and hence it is the only inverse of $a$. Therefore, by injectivity $\phi(a^{-1}) = \phi(a)^{-1}$.

\subsection*{Exercise 2.3.13}
This proof of Lemma 2.3.12 is quite abstract. Find
a more concrete proof, taking equation (2.2) as your definition of
characteristic. (You will still need the fact that $\phi$ is injective.)
\\\\
\textbf{Solution:}
By (2.2) we have:

\[ char R = 
    \begin{cases} 
        \text{least}\ n>0: n*1_R = 0_R & ,\ \text{if such an n exists} \\
        0 & ,\ \text{otherwise} \\
    \end{cases}
\]
\\
We know that $\phi(1_K) = 1_L$ and $\phi(0_K) = 0_L$, since $\phi$ is injective, then also
$\phi(n\cdot 1_K) = n \cdot 1_L \ \forall n \in \mathbb{N}$. We have two possible cases for the characteristic $c$ of $K$ ($char K$),
$c = 0$ or $c > 0$.\\\\
If $c=0$, then $\phi(0_K) = 0_L = 0$. Therefore $char L = c = char K$.\\
If $c>0$, then $\phi(c \cdot 1_K) = c\cdot 1_L = 0$. Therefore $char L = c = char K$.

\subsection*{Exercise 2.3.15}
What is the prime subfield of $\mathbb{R}$? Of $\mathbb{C}$?
\\\\
\textbf{Solution:}
For $\mathbb{R}$ it is $\mathbb{Q}$. For $\mathbb{C}$ it is also $\mathbb{Q}$.
See \textbf{Lemma 2.3.16}.

\subsection*{Exercise 2.3.25}
What are the irreducible elements of a field?
\\\\
\textbf{Solution:}
We know thar for a ring $R$, $r$ is irreducible if $r$ is not 0 or a unit and if for $a,b \in R$, 
then $r = ab \implies$ $a$ or $b$ is a unit.
However, we know that every element of a field $K$ is a either a unit or 0. Therefore, there are no
irreducible elements in a field.

\section*{Chapter 3. Polynomials}

\subsection*{Exercise 3.1.4}
Show that whenever $R$ is a finite nontrivial ring, it is
possible to find distinct polynomials over $R$ that induce the same function
$R \rightarrow R$. (Hint: are there finitely or infinitely many polynomials
over $R$? Functions $R \rightarrow R$?)
\\\\
\textbf{Solution:}

\subsection*{Exercise 3.1.8}
What happens to everything in the previous paragraph
if we substitute $t=u^2+c$ instead?
\\\\
\textbf{Solution:}

\subsection*{Exercise 3.1.13}
Let $p$ be a prime and consider the field $\mathbb{F}_p(t)$ of
rational expressions over $\mathbb{F}_p$?. Show that $t$ has no $p$th root in $\mathbb{F}_p(t)$.
(Hint: consider degrees of polynomials.)
\\\\
\textbf{Solution:}

\subsection*{Exercise 3.2.4}
Prove that the ideals in Warning 3.2.3 are indeed not
principal.
\\\\
\textbf{Solution:}

\subsection*{Exercise 3.3.5}
If I gave you a quadratic over $\mathbb{Q}$, how would you decide
whether it was reducible or irreducible?
\\\\
\textbf{Solution:}

\subsection*{Exercise 3.3.13}
The last step in (3.9) was $'deg(\bar h) \leq deg(h)'$. Why
is that true? And when does equality hold?
\\\\
\textbf{Solution:}

\subsection*{Exercise 3.3.15}
Use Eisenstein's criterion to show that for every
$n \ge 1$, there is an irreducible polynomial over $\mathbb{Q}$ of degree $n$.
\\\\
\textbf{Solution:}

\section*{Chapter 4. Field extensions}

\subsection*{Exercise 4.1.3}
Find two examples of fields $K$ such that $Q \subsetneqq K \subsetneqq \mathbb{Q}(\sqrt2, i)$ 
\\\\
\textbf{Solution:} $K=\mathbb{Q}(\sqrt2) = \{a+b\sqrt2: a,b \in \mathbb{Q}\}$ and \\ $K=\mathbb{Q}(i)= \{a+bi: a,b \in \mathbb{Q}\}$

\subsection*{Exercise 4.1.5}
Check the truth of all the statements in the previous paragraph.
\\\\
\textbf{Solution:} Follow trivially from definitions of interesection and subfields. See \textbf{Lemma 2.2.3} for showing interesection of subfields still remains a subfield.

\subsection*{Exercise 4.1.7}
What is the subfield of $\mathbb{C}$ generated by $\{7/8\}$? By $\{2+3i\}$? By $\mathbb{R}\cup\{i\}$?
\\\\
\textbf{Solution:}
Since $\mathbb{C}$ is of characteristic $0$, by \textbf{Lemma 2.3.16} the prime subfield of $\mathbb{C}$ is $\mathbb{Q}$.
Since $\mathbb{Q}$ contains $\{7/8\}$ and by definition of prime subfield, it is the intersection of all the subfields of $\mathbb{C}$ containing $\{7/8\}$, 
hence $\mathbb{Q}$ is generated by $\{7/8\}$.\\\\
Let $L$ be the subfield of $\mathbb{C}$ generated by $\{2+3i\}$. Then $L = \{2a+3bi: a,b \in \mathbb{Q}\}$ by similar argument as \textbf{Example 4.1.6 (ii)}.\\\\
Similarly, let $L$ be the subfield of $\mathbb{C}$ generated by $ \mathbb{R}\cup\{i\}$. Then \\ $L=\mathbb{R} \cup \{a+bi: a,b \in \mathbb{Q}\} \stackrel{?}{=} \{a+bi: a \in \mathbb{R}, b \in \mathbb{Q}\}$.

\subsection*{Exercise 4.1.11}
Let $M:K$ be a field extension. Show that $K(Y \cup Z) = (K(Y))(Z)$ whenever $Y,Z \subseteq M$. 
\\\\
\textbf{Solution:}

\subsection*{Exercise 4.2.2}
Show that every element of $K$ is algebraic over $K$.
\paragraph{Solution:}
Since $K$ is a field, $\forall k \in K: \exists -k \in K: k + (-k) = (-k) + k = 0$.
Therefore, $\forall k \in K$, we can choose $f(t) = t - k \in K[t]$. Hence we have that $f \neq 0$
and $f(k) = k - k = 0$. Therefore $\forall k \in K, k$ is algebraic over $K$.


\subsection*{Exercise 4.2.9}
What is the minimal polynomial of an element of $K$?
\paragraph{Solution:}
We can refer back to \textbf{Exercise 4.2.2}.
If we let $m(t) = t - k$, then we see that it is indeed monic and unique $\forall k \in K$ satisfying condition (4.2).

\subsection*{Exercise 4.3.5}
Let $M:K$ and $L:K$ be field extensions, and let $\phi: M \rightarrow L$ be a homomorphism over $K$. Show that if $\alpha \in M$ has
minimal polynomial $m$ over $K$ then $\phi(\alpha) \in L$ also has minimal polynomial $m$ over $K$.
\\\\
\textbf{Solution:}

\subsection*{Exercise 4.3.9}
Fill in the details of the last paragraph of that proof.
\paragraph{Solution:}
We show that there is at most one homomorphism $\phi: K(t) \rightarrow L$ over $K$ such that $\phi(t)= \beta$.
We let $\phi$ and $\phi'$ be two such homomorphisms. Then we have that $\phi(t) = \beta = \phi'(t)$. By \textbf{Lemma 4.3.1 (ii)}
we have that $t$ generates $K(t)$ over $K$, and hence by \textbf{Lemma 4.3.6} $\phi = \phi' \quad \square$


\subsection*{Exercise 4.3.15}
Prove that $\mathbb{Q}(\sqrt 2, \sqrt 3) = \mathbb{Q}(\sqrt 2 + \sqrt 3)$.
\\\\
\textbf{Solution:}
We know that $\sqrt2 + \sqrt3 \in \mathbb{Q}(\sqrt2, \sqrt3)$ and hence $\mathbb{Q}(\sqrt2 + \sqrt3) \subseteq \mathbb{Q}(\sqrt2 , \sqrt3)$. Now we show the inclusion the other way.
We use the hint and get that $(\sqrt2 + \sqrt3)^3 = 11\sqrt2 + 9\sqrt3 \in \mathbb{Q}(\sqrt2, \sqrt3)$. Then we have that:\\ $11\sqrt2 + 9\sqrt3 - 9(\sqrt2 + \sqrt3 ) = 2\sqrt2 \in \mathbb{Q}(\sqrt2, \sqrt3)$, 
hence $\sqrt2 \in \mathbb{Q}(\sqrt2, \sqrt3)$. Similarly, we get that $\sqrt3 \in \mathbb{Q}(\sqrt2, \sqrt3)$. Therefore, $\mathbb{Q}(\sqrt2, \sqrt3) \subseteq \mathbb{Q}(\sqrt2 + \sqrt3) \quad \square$


\subsection*{Exercise 4.3.18}
How many elements does the field $\mathbb{F}_3(\sqrt 2)$ have? What about $\mathbb{F}_2(\alpha)$, where $\alpha$ is a root of $1+t+t^2$?
\\\\
\textbf{Solution:}
We know that $\mathbb{F}_3(\sqrt 2)$ can be constructed as $\mathbb{F}_3[t]/\langle t^2 - 2 \rangle$. Hence, any element of the field has the form
$a_0 + a_1t + \langle t^2 - 2\rangle$ with $a_i \in \mathbb{F}_3$. Hence, there are $3^2 = 9$ elements.
\\\\
In a similar manner, we know that $\mathbb{F}_2(\alpha)$ can be constructed as $\mathbb{F}_2[t]/\langle t^2+t+1 \rangle$. Hence any element of the field has the form
$a_0 + a_1t + \langle t^2+t+1\rangle$ with $a_i \in \mathbb{F}_2$. Hence there are $2^2 = 4$ elements.

\section*{Chapter 5. Degree}
\subsection*{Exercise 5.1.9}
Write out the addition and multiplication tables of $\mathbb{F}_2(\alpha)$.
\\\\
\textbf{Solution:}
The tables are straightforward, using modulo arithmetic and the irreducible polynomial evaluated at $\alpha$.\\
    \begin{center}
        \renewcommand\arraystretch{1.3}
        \setlength\doublerulesep{0pt}
        \begin{tabular}{r||*{4}{2|}}
            + & 0 & 1 & \alpha & 1+\alpha \\
            \hline\hline
            0 & 0 & 1 & \alpha & 1+\alpha \\ 
            \hline
            1 & 1 & 0 & 1+\alpha & \alpha \\ 
            \hline
            $\alpha$ & \alpha & 1+\alpha & 0 & 1 \\ 
            \hline
            $1+\alpha$ & 1+\alpha & \alpha & 1 & 0 \\ 
            \hline
        \end{tabular}
    \end{center}
    \vspace{0.5 cm}
    \begin{center}
        \renewcommand\arraystretch{1.3}
        \setlength\doublerulesep{0pt}
        \begin{tabular}{r||*{4}{2|}}
            $\times$ & 0 & 1 & \alpha & 1+\alpha \\
            \hline\hline
            0 & 0 & 0 & 0 & 0 \\ 
            \hline
            1 & 0 & 1 & \alpha & 1+\alpha \\ 
            \hline
            $\alpha$ & 0 & \alpha & 1+\alpha & 1\\ 
            \hline
            $1+\alpha$ & 0 & 1+\alpha & 1 & \alpha \\ 
            \hline
        \end{tabular}
    \end{center}

\subsection*{Exercise 5.1.13}
Give an example of to show that the inequalityin Corollary 5.1.12 can be strict. Your example can be as trivial as you like.
\\\\
\textbf{Solution:}
We choose our fields and hence extensions to be $\mathbb{C}:\mathbb{R}:\mathbb{Q}$. We also choose $\beta = \sqrt2 \in \mathbb{C}$.
The minimal polynomial of $\sqrt2$ over $\mathbb{Q}$ is $m = t^2 - 2$, 
then deg$_\mathbb{Q}(\beta) = [\mathbb{Q}(\beta):\mathbb{Q}] = 2$.\\\\
Similarly, the minimal polynomial of $\sqrt2$ over $\mathbb{R}$ is $m = t - \sqrt2$, 
then deg$_\mathbb{R}(\beta) = [\mathbb{R}(\beta):\mathbb{R}] = 1$.\\\\
Hence we have that $[\mathbb{R}(\beta):\mathbb{R}] < [\mathbb{Q}(\beta):\mathbb{Q}] \quad \square$

\subsection*{Exercise 5.1.16}
Let $M:K$ be a field extension and $\alpha$ a transcendental element of $M$. Can every element of $K(\alpha)$ be represented
as a polynomial in $\alpha$ over $K$?
\\\\
\textbf{Solution:}
We have that $K(\alpha) = \left\{ \frac {f(\alpha)}{g(\alpha)} : f,g \in F[t]\right\}$, which is just $K(t)$, the field
rational expressions. Therefore it is not polynomial is $\alpha$ over $K$. 

\subsection*{Exercise 5.1.20}
Show that a field extension whose degree is a prime number must be simple.
\\\\
\textbf{Solution:}
Let $M:K(\alpha):K$ be field extensions where $M$ and $K$ are arbitrary fields, $\alpha \in M$, and $[M:K] = p$, where $p$ is prime. By \textbf{Theorem 5.1.17 (iii)} we have $[M:K] = [M:K(\alpha)][K(\alpha):K]$.
Hence, we must have that $[K(\alpha):K] = 1$ or $p$,
however, we also know that $K(\alpha) \neq K$, hence $[K(\alpha):K] = p$, and therefore,
$[M:K(\alpha)]=1$, which by \textbf{Example 5.1.3} tells us $M=K(\alpha)$.
Hence $M:K$ is a simple.

\subsection*{Exercise 5.1.23}
Generalize Example 5.1.22. In other words, what
general result does the argument of Example 5.1.22 prove, not involving
the particular numbers chosen there?
\\\\
\textbf{Solution:}
Let $M:K$ be a field extension and $\alpha_1,...,\alpha_n \in M$. If $gcd(deg_K(\alpha_1)\\,...,deg_K(\alpha_n))=1$ (i.e., coprime),
then we have that, $[K(\alpha_1,...,\alpha_n):K] = [K(\alpha_1):K]...[K(\alpha_n):K]$ 

\subsection*{Exercise 5.2.5}
Let $M:K$ be a field extension and $K \subseteq L \subseteq M$. In the proof of Proposition 5.2.4, I said that if $L$ is a
subfield of $M$ then $L$ is a $K$-linear subspace of $M$. Why is that true? And is the converse also true? Give proof or a counterexample.  
\\\\
\textbf{Solution:}
We know that $M$ acts as a vector space over $K$. If $L$ is a subfield of $M$, then we can similarly conclude that $L$ acts as a vector space over $K$.
Since we have that $L$ is a subset of $M$ (a subfield) we can conclude that $L$ is a linear ($K$-linear) subspace of $M$ (by definition of a linear subspace).
\\\\ 
The converse is not true.

\subsection*{Exercise 5.2.8}
Let $M:K$ be a field extension and write $L$ for the set of elements of $M$ algebraic over $K$. By imitating the proof of Proposition 5.2.7,
prove that $L$ is a subfield of $M$.
\\\\
\textbf{Solution:}
We have that $L = \{\alpha \in M: [K(\alpha):K]<\infty\}$.\\
Then $\forall \alpha,\beta \in L,\ [K(\alpha,\beta):K] \leq [K(\alpha):K][K(\beta):K] < \infty$\\
Now $\alpha+\beta \in K(\alpha,\beta)$, so $K(\alpha+\beta) \subseteq K(\alpha,\beta)$, hence \\
$[K(\alpha+\beta):K] \leq [K(\alpha,\beta):K]<\infty$, giving $\alpha+\beta \in L$. Similarly, $\alpha \cdot \beta \in L$.\\
Then $\forall \alpha \in L,\ [K(-\alpha):K] = [K(\alpha):K]<\infty$, giving $-\alpha \in L$.
Similarly, $1/\alpha \in L$ (if $\alpha \neq 0$), and clearly $0,1 \in L \quad \square$

\subsection*{Exercise 5.3.7}
Find an example of Lemma 5.3.6 where $[LL':L]=2$, and another where $[LL':L]=1$.
\\\\
\textbf{Solution:}
If we let $L = \mathbb{Q}(\sqrt2)$ and $L' = \mathbb{Q}(\sqrt3)$, we then get $LL' = \mathbb{Q}(\sqrt2,\sqrt3)$.
Then $[\mathbb{Q}(\sqrt2,\sqrt3):\mathbb{Q}(\sqrt3)] = 2$.\\\\
If we let $L = \mathbb{Q}(\sqrt4)$ and $L' = \mathbb{Q}(\sqrt3)$, we then get $LL' = \mathbb{Q}(\sqrt4,\sqrt3)$.
Then $[\mathbb{Q}(\sqrt4,\sqrt3):\mathbb{Q}(\sqrt3)] = 1$.
\end{document}

\section*{Chapter 6. Splitting fields}
\subsection*{Exercise 5.1.9}
Write out the addition and multiplication tables of $\mathbb{F}_2(\alpha)$.
\\\\
\textbf{Solution:}