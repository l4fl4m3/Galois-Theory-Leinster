\documentclass{article}
\usepackage{amsfonts} 
\usepackage{amsmath}
\usepackage{amssymb}
\usepackage{dcolumn}
\usepackage{tikz-cd}


\newcolumntype{2}{D{.}{}{4.0}}
\title{Solutions: Galois Theory by Tom Leinster}
\author{Hassaan Naeem}
\date{\today}
\begin{document}
\maketitle
\section*{Chapter 1. Overview of Galois Theory}
\subsection*{Exercise 1.1.3} 
Both proofs of ‘if’ contain little gaps: ‘It follows by
induction’ in the first proof, and ‘it’s easy to see’ in the second. Fill
them.\

\paragraph{Solution:} We show both both parts (i) and (ii) seperately

\begin{enumerate}
    \item[(i)] Follows from induction that for any polynomial $p$ over $\mathbb{R}$, $\overline{ p(w)} = p(\overline w)$:\\
        \linebreak 
        Let $p(w) = c_0 + c_1w^1 + c_2w^2 + ... + c_nw^n$ where $w^n \in \mathbb{C}$ and $c_n \in \mathbb{C}$. \\
            \begin{equation*}
                \begin{aligned}
                    \overline{p(w)} & = \overline{c_0 +c_1w^1 + c_2w^2 + ... + c_nw^n}\\
                    & = \overline{c_0} + \overline{c_1w^1} + \overline{c_2w^2} + ... + \overline{c_nw^n}\\
                    & = c_0 + c_1\overline{w^1} + c_2\overline{w^2} + ... + c_n\overline{w^n}\\
                    & = p(\overline w)
                \end{aligned}
            \end{equation*}
    \item[(ii)] Checking that r is the zero polynomial:\\
        \linebreak
        \textbf{Lemma}. If $r(x) = a_0 + a_1x^1+ ... + a_nx^n $ and $r(x) = 0,   \forall x \neq 0 $, then $a_0 = 0$. Since $\mathbb{Q}$ is a field, and must contain a zero.\\
        
        By this lemma we have that $\forall x$, $ r(x) = 0$ and therefore\\
        $r(x) = 0 = x(a_1+ a_2x + ... + a_nx^{n-1}) = a_1+ a_2x + ... + a_nx^{n-1}$ and $\forall x \neq 0$ $a_1 = 0$.
        Hence, we repeat the Lemma and show that all $a_1, ..., a_n = 0$. Therefore $r(x)$ is the zero polynomial.
\end{enumerate}

\subsection*{Exercise 1.1.6} 
Let $z \in \mathbb{Q}$. Show that $z$ is not conjugate to $z'$ for any complex number $z' \neq z$.

\paragraph{Solution:}

\subsection*{Exercise 1.1.10}
Suppose that $(z_1, ..., z_k)$ and $(z_1', ..., z_k')$ are conjugate. Show that $z_i$ and $z_i'$ are conjugate, for each $i \in \{1, ..., k\}$

\paragraph{Solution:} By \textbf{Definition 1.1.9} have that:
    \begin{equation*}
        \begin{aligned}
            p(z_1, ..., z_k) & = 0 \Longleftrightarrow p(z_1', ..., z_k')=0
        \end{aligned}
    \end{equation*}
\noindent
When $k=1$ we have that:
\\\\
\centerline{$p(z_1)=0 \Longleftrightarrow p(z_1') = 0 \implies$  $z_1$ and $z_1'$ are conjugate}
\\\\
Similarly, for any $k$:
\\\\
\centerline{$p(z_i)=0 \Longleftrightarrow p(z_i') = 0 \implies$  $z_i$ and $z_i'$ are conjugate}

\subsection*{Exercise 1.2.2}
Show that $Gal(f)$ is a subgroup of $S_k$.


\section*{Chapter 2. Group actions, rings and fields}

\subsection*{Exercise 2.1.3}
Check that $\bar g$ is a bijection for each $g \in G$. Also check that $\Sigma$ is a homomorphism.

\paragraph{Solution:} We show injectivity (i), surjectivity (ii) and homomorphism (iii) :

    \begin{enumerate}
        \item[(i)] Injectivity:\\
            \linebreak
            Let $x,y \in X$ and  $\bar g$ be our bijection\\
            If we have $\bar g(x) = \bar g(y)$
                \begin{equation*}
                    \begin{aligned}
                        \leadsto gx &= gy \leadsto g^{-1}(gx) = g^{-1}(gy) \leadsto (g^{-1}g)x = (g^{-1}g)y \leadsto ex = ey \leadsto x=y\quad \square\\
                    \end{aligned}
                \end{equation*}
        \item[(ii)] Surjectivity:\\
            \linebreak
            We know that $f: X \rightarrow Y$ is surjective iff $\forall y \in Y\ \exists x \in X: f(x)=y$ \\
            We let $x \in X$ and $e$ be the identity in $G$ then, \\
            $x = ex = (gg^{-1})x = g(g^{-1}x) = gy = \bar g(y)$ where $ y = g^{-1}x \in X \quad \square$ \\ 
            \\Therefore $\bar g$ is both injective and surjective, hence bijective. \\
        \item[(i)] $\Sigma$ is Homomorphism:\\
            \linebreak
            We have the map:
            \begin{equation*}
                \begin{aligned}
                     \Sigma:\ &G \rightarrow Sym(X)\\
                              & g \mapsto \bar g
                \end{aligned}
            \end{equation*}
            We know that $\bar g$ is well defined\\
            Then we take $g, h \in G, x \in X$, then by \textbf{Definition 2.1.1}.\\
            \begin{equation*}
                \begin{aligned}
                    \Sigma(gh)(x)& = gh(x) = g(hx)\\
                                 & = \Sigma(g)(\Sigma(h)(x)) = \Sigma(g) \circ \Sigma(h) \ (x) \quad \square
                \end{aligned}
            \end{equation*}
        \end{enumerate}

\subsection*{Exercise 2.1.10}
Example \textbf{2.1.9(iii)} shows that the action of the isometry
cube $G$ of the cube on the set $X$ of long diagonals is not faithful.
By \textbf{Lemma 2.1.8}, there must be some non-identity isometry of the
cube that fixes all four long diagonals. In fact, there is exactly one.
What is it?

\paragraph{Solution:} We show both both parts (i) and (ii) seperately


\subsection*{Exercise 2.2.6}
Prove that the only subring of a ring $R$ that is also an
ideal is $R$ itself.

\paragraph{Solution:} We know that $I$ is an ideal of $R$ if:\\
    \begin{equation*}
        \begin{aligned}
            &(I, +) \le (R, +)\quad  \text{[I is additive subgroup of R]}\\
            & \&\ \forall r \in R, x \in I:\\
            & (1)\ r \cdot x \in I\\
            & (2)\ x \cdot r \in I
        \end{aligned}
    \end{equation*}

    We know that a subring $S$ of $R$ is a subset $S \subseteq R$ containing $0$ and $1$.\\
    \\Therefore if we take $S$ to be an ideal aswell then:
    \begin{equation*}
        \begin{aligned}
            &(S, +) \le (R, +)\\
            & \&\ \forall r \in R, s \in S:\\
            & (1)\ r \cdot s \in S\\
            & (2)\ s \cdot r \in S
        \end{aligned}
    \end{equation*}
    But we know that $1 \in S$. Therefore, $\forall r \in R,$ $(1)\ 1 \cdot r = r \in S$ and $(2)\ r \cdot 1 = r \in S$\\
    Therefore $\forall r \in R, r \in S \implies S = R \quad \square$

\subsection*{Exercise 2.2.8}
The trivial ring or zero ring is the one-element set
with its only possible ring structure. Show that the only ring in which
$0 = 1$ is the trivial ring.

\paragraph{Solution:} Let $(R, +, \cdot)$ be our commutative, unital ring.\\
If $1=0$ in $R$, then $\forall r \in R$ we have $r= 1r = 0r = 0\quad \square$

\subsection*{Exercise 2.2.8}
Fill in the details of Example 2.2.13.

\paragraph{Solution:} We suppose that $I \subseteq \mathbb{Z}$ is an ideal and we take $n \in I$ to 
be the least positive integer in $I$. We have obviously that $\langle n \rangle \subseteq I$. Then we assume that
that $m \in I$, by the division algorithm we know that:

    \begin{equation*}
        \begin{aligned}
            m &= qn +r  \quad &(0 \leq r < n)\\
            r &= m - qn \quad &\in I
        \end{aligned}
    \end{equation*}
Therefore $r=0$ $\leadsto m = qn$. Therefore $m \in \langle n \rangle$ and we have that $I \subseteq \langle n \rangle$.
Hence we have equality, $I = \langle n \rangle \quad \square$


\subsection*{Exercise 2.2.15}
Let $r$ and $s$ be elements of an integral domain.\\
Show that $r | s |r \Longleftrightarrow \langle r \rangle = \langle s \rangle \Longleftrightarrow s = ur$ for some unit $u$.
\paragraph{Solution:}
If we have that $r|s|r$ then $\exists\ a \in R: s = ar$ and $ \exists\ b \in R: r = bs$ then:
    \begin{equation*}
        \begin{aligned}
            \frac{s}{a} &= bs\\
            b &= \frac{1}{a} \leadsto ab=1 \leadsto b = a^{-1}
        \end{aligned}
    \end{equation*}
Then we have that $s=ar$, and we have just shown that a is a unit, hence $s=ur$. Therefore $r|s|r \implies s = ur$
\\\\
If we have $\langle r \rangle = \langle s \rangle$, then $r=s$. Hence, 
\begin{equation*}
    \begin{aligned}
        r = 1s \quad &\&\quad s = 1r\\
        r = as \quad &\&\quad s = ar\quad (\text{where}\ a=1)\\
        \implies s|r  \ &\&\quad r|s
    \end{aligned}
\end{equation*}
Therefore $\langle r \rangle = \langle s \rangle \implies r|s|r$
\\\\
If we have that $s=ur$ for some unit $u$, then also we have that
\begin{equation*}
    \begin{aligned}
        &u^{-1}s = u^{-1}ur \leadsto r = u^{-1}s\\
        &\text{Therefore}\ s \in \langle r \rangle\ \&\ r \in \langle s \rangle, \langle s \rangle \subseteq \langle r \rangle\ \&\ \langle r \rangle \subseteq \langle s \rangle \\
        &\implies  \langle r \rangle = \langle s \rangle
    \end{aligned}
\end{equation*}
Therefore $s=ur \implies \langle r \rangle = \langle s \rangle$

\subsection*{Exercise 2.3.1}
Write down all the examples of fields that you know.
\\\\
\textbf{Solution:}
$\mathbb{C}, \mathbb{R}, \mathbb{Q}$

\subsection*{Exercise 2.3.5}
Let $\phi : K \rightarrow L$ be a homomorphism of fields and let
$0 \neq a \in K$. Prove that $\phi(a^{-1}) = \phi(a)^{-1}$.
Why is $\phi(a)^{-1}$ defined?
\\\\
\textbf{Solution:}
Since $K$ is a field, and the fact that $0 \neq a \in K$, we have that $a$ is a unit, $aa^{-1}=1$, and $a^{-1} \in K$.
By \textbf{Lemma 2.3.3}, we have that $\phi: K \rightarrow L$ is injective. Hence, $\phi(a)\phi(a^{-1}) = \phi(a \circ a^{-1}) = \phi(1) = 1$, and
$\phi(a^{-1})\phi(a) = \phi(a^{-1} \circ a) = \phi(1) = 1$. Therefore we have that $\phi(a^{-1})$ is both a left and right inverse
of $a$ and hence it is the only inverse of $a$. Therefore, by injectivity $\phi(a^{-1}) = \phi(a)^{-1}$.

\subsection*{Exercise 2.3.13}
This proof of Lemma 2.3.12 is quite abstract. Find
a more concrete proof, taking equation (2.2) as your definition of
characteristic. (You will still need the fact that $\phi$ is injective.)
\\\\
\textbf{Solution:}
By (2.2) we have:

\[ char R = 
    \begin{cases} 
        \text{least}\ n>0: n*1_R = 0_R & ,\ \text{if such an n exists} \\
        0 & ,\ \text{otherwise} \\
    \end{cases}
\]
\\
We know that $\phi(1_K) = 1_L$ and $\phi(0_K) = 0_L$, since $\phi$ is injective, then also
$\phi(n\cdot 1_K) = n \cdot 1_L \ \forall n \in \mathbb{N}$. We have two possible cases for the characteristic $c$ of $K$ ($char K$),
$c = 0$ or $c > 0$.\\\\
If $c=0$, then $\phi(0_K) = 0_L = 0$. Therefore $char L = c = char K$.\\
If $c>0$, then $\phi(c \cdot 1_K) = c\cdot 1_L = 0$. Therefore $char L = c = char K$.

\subsection*{Exercise 2.3.15}
What is the prime subfield of $\mathbb{R}$? Of $\mathbb{C}$?
\\\\
\textbf{Solution:}
For $\mathbb{R}$ it is $\mathbb{Q}$. For $\mathbb{C}$ it is also $\mathbb{Q}$.
See \textbf{Lemma 2.3.16}.

\subsection*{Exercise 2.3.25}
What are the irreducible elements of a field?
\\\\
\textbf{Solution:}
We know thar for a ring $R$, $r$ is irreducible if $r$ is not 0 or a unit and if for $a,b \in R$, 
then $r = ab \implies$ $a$ or $b$ is a unit.
However, we know that every element of a field $K$ is a either a unit or 0. Therefore, there are no
irreducible elements in a field.

\section*{Chapter 3. Polynomials}

\subsection*{Exercise 3.1.4}
Show that whenever $R$ is a finite nontrivial ring, it is
possible to find distinct polynomials over $R$ that induce the same function
$R \rightarrow R$. (Hint: are there finitely or infinitely many polynomials
over $R$? Functions $R \rightarrow R$?)l
\\\\
\textbf{Solution:}

\subsection*{Exercise 3.1.8}
What happens to everything in the previous paragraph
if we substitute $t=u^2+c$ instead?
\\\\
\textbf{Solution:}

\subsection*{Exercise 3.1.13}
Let $p$ be a prime and consider the field $\mathbb{F}_p(t)$ of
rational expressions over $\mathbb{F}_p$. Show that $t$ has no $p$th root in $\mathbb{F}_p(t)$.
(Hint: consider degrees of polynomials.)
\\\\
\textbf{Solution:}
A rational expression over $K$ is $\frac{f(t)}{g(t)}$ where $f(t), g(t) \in K[t]$ with $g \neq 0$.
For any $\frac{f(t)}{g(t)} \in \mathbb{F}_p(t)$ where $f(t), g(t) \in \mathbb{F}_p[t]$, suppose we have
have that $\left( \frac{f(t)}{g(t)} \right)^p = t$. We then have that $f^p = tg^p$. Then $deg(f^p) = np$ where $n=deg(f)$ and
$deg(tg^p) = deg(t)+deg(g^p) = 1 + mp$ where $m=deg(g)$, hence we have $np = mp+1 \leadsto p = \frac{1}{n-m}$.
But this is impossible since $p$ is prime, hence a contradiction, hence $t$ has no $p$th root in $\mathbb{F}_p(t)$.

\subsection*{Exercise 3.2.4}
Prove that the ideals in Warning 3.2.3 are indeed not
principal.
\\\\
\textbf{Solution:}

\subsection*{Exercise 3.3.5}
If I gave you a quadratic over $\mathbb{Q}$, how would you decide
whether it was reducible or irreducible?
\\\\
\textbf{Solution:}
By \textbf{Lemma 3.3.1 (ii)}, if the quadratic has a root in $\mathbb{Q}$, then it is reducible.
By the same lemma \textbf{(iii)}, if the quadratic has no root in $\mathbb{Q}$, then it is irreducible. 

\subsection*{Exercise 3.3.13}
The last step in (3.9) was $'deg(\bar h) \leq deg(h)'$. Why
is that true? And when does equality hold?
\\\\
\textbf{Solution:}
$\bar h = hmodp$. Therefore if $p|a_{n_h}$ then $a_{n_{\bar h}}=0$ and $deg(\bar h) < deg(h)$. If $p \nmid a_{n_h}$ then $a_{n_h} = a_{n_{\bar h}}$ and $deg(\bar h)= deg(h)$.
Therefore $deg(\bar h) \leq deg(h)$. Equality holds on the preceding condition.

\subsection*{Exercise 3.3.15}
Use Eisenstein's criterion to show that for every
$n \ge 1$, there is an irreducible polynomial over $\mathbb{Q}$ of degree $n$.
\\\\
\textbf{Solution:}
Let $f(t) = a_0 + ... + a_nt^n \in \mathbb{Q}[t]$ with $n \geq 1$.
For $n \geq 1$, we can always choose an $f \in \mathbb{Q}[t]$ such that $f(t)=a_nt^n + a_0$, and we can further
always choose an $a_n, a_0$ and $p$ such that $p \nmid a_n,\ p|a_0,\ p^2 \nmid a_0$. Hence, we have $f(t) = a_nt^n + a_0$ fulfilling the
Eisenstein criterion, and hence $f(t)$ is irreducible over $\mathbb{Q}$. As an example, we can always choose
$f(t) = t^n +2$ and $p=2$.

\section*{Chapter 4. Field extensions}

\subsection*{Exercise 4.1.3}
Find two examples of fields $K$ such that $Q \subsetneqq K \subsetneqq \mathbb{Q}(\sqrt2, i)$ 
\\\\
\textbf{Solution:} $K=\mathbb{Q}(\sqrt2) = \{a+b\sqrt2: a,b \in \mathbb{Q}\}$ and \\ $K=\mathbb{Q}(i)= \{a+bi: a,b \in \mathbb{Q}\}$

\subsection*{Exercise 4.1.5}
Check the truth of all the statements in the previous paragraph.
\\\\
\textbf{Solution:} Follow trivially from definitions of interesection and subfields. See \textbf{Lemma 2.2.3} for showing interesection of subfields still remains a subfield.

\subsection*{Exercise 4.1.7}
What is the subfield of $\mathbb{C}$ generated by $\{7/8\}$? By $\{2+3i\}$? By $\mathbb{R}\cup\{i\}$?
\\\\
\textbf{Solution:}
Since $\mathbb{C}$ is of characteristic $0$, by \textbf{Lemma 2.3.16} the prime subfield of $\mathbb{C}$ is $\mathbb{Q}$.
Since $\mathbb{Q}$ contains $\{7/8\}$ and by definition of prime subfield, it is the intersection of all the subfields of $\mathbb{C}$ containing $\{7/8\}$, 
hence $\mathbb{Q}$ is generated by $\{7/8\}$.\\\\
Let $L$ be the subfield of $\mathbb{C}$ generated by $\{2+3i\}$. Then $L = \{2a+3bi: a,b \in \mathbb{Q}\}$ by similar argument as \textbf{Example 4.1.6 (ii)}.\\\\
Similarly, let $L$ be the subfield of $\mathbb{C}$ generated by $ \mathbb{R}\cup\{i\}$. Then \\ $L=\mathbb{R} \cup \{a+bi: a,b \in \mathbb{Q}\} \stackrel{?}{=} \{a+bi: a \in \mathbb{R}, b \in \mathbb{Q}\}$.

\subsection*{Exercise 4.1.11}
Let $M:K$ be a field extension. Show that $K(Y \cup Z) = (K(Y))(Z)$ whenever $Y,Z \subseteq M$. 
\\\\
\textbf{Solution:}

\subsection*{Exercise 4.2.2}
Show that every element of $K$ is algebraic over $K$.
\paragraph{Solution:}
Since $K$ is a field, $\forall k \in K: \exists -k \in K: k + (-k) = (-k) + k = 0$.
Therefore, $\forall k \in K$, we can choose $f(t) = t - k \in K[t]$. Hence we have that $f \neq 0$
and $f(k) = k - k = 0$. Therefore $\forall k \in K, k$ is algebraic over $K$.


\subsection*{Exercise 4.2.9}
What is the minimal polynomial of an element of $K$?
\paragraph{Solution:}
We can refer back to \textbf{Exercise 4.2.2}.
If we let $m(t) = t - k$, then we see that it is indeed monic and unique $\forall k \in K$ satisfying condition (4.2).

\subsection*{Exercise 4.3.5}
Let $M:K$ and $L:K$ be field extensions, and let $\phi: M \rightarrow L$ be a homomorphism over $K$. Show that if $\alpha \in M$ has
minimal polynomial $m$ over $K$ then $\phi(\alpha) \in L$ also has minimal polynomial $m$ over $K$.
\\\\
\textbf{Solution:}

\subsection*{Exercise 4.3.9}
Fill in the details of the last paragraph of that proof.
\paragraph{Solution:}
We show that there is at most one homomorphism $\phi: K(t) \rightarrow L$ over $K$ such that $\phi(t)= \beta$.
We let $\phi$ and $\phi'$ be two such homomorphisms. Then we have that $\phi(t) = \beta = \phi'(t)$. By \textbf{Lemma 4.3.1 (ii)}
we have that $t$ generates $K(t)$ over $K$, and hence by \textbf{Lemma 4.3.6} $\phi = \phi' \quad \square$


\subsection*{Exercise 4.3.15}
Prove that $\mathbb{Q}(\sqrt 2, \sqrt 3) = \mathbb{Q}(\sqrt 2 + \sqrt 3)$.
\\\\
\textbf{Solution:}
We know that $\sqrt2 + \sqrt3 \in \mathbb{Q}(\sqrt2, \sqrt3)$ and hence $\mathbb{Q}(\sqrt2 + \sqrt3) \subseteq \mathbb{Q}(\sqrt2 , \sqrt3)$. Now we show the inclusion the other way.
We use the hint and get that $(\sqrt2 + \sqrt3)^3 = 11\sqrt2 + 9\sqrt3 \in \mathbb{Q}(\sqrt2+ \sqrt3)$. Then we have that:\\ $11\sqrt2 + 9\sqrt3 - 9(\sqrt2 + \sqrt3 ) = 2\sqrt2 \in \mathbb{Q}(\sqrt2 + \sqrt3)$, 
hence $\sqrt2 \in \mathbb{Q}(\sqrt2 + \sqrt3)$. Similarly, we get that $\sqrt3 \in \mathbb{Q}(\sqrt2 + \sqrt3)$. Therefore, $\mathbb{Q}(\sqrt2, \sqrt3) \subseteq \mathbb{Q}(\sqrt2 + \sqrt3) \quad \square$


\subsection*{Exercise 4.3.18}
How many elements does the field $\mathbb{F}_3(\sqrt 2)$ have? What about $\mathbb{F}_2(\alpha)$, where $\alpha$ is a root of $1+t+t^2$?
\\\\
\textbf{Solution:}
We know that $\mathbb{F}_3(\sqrt 2)$ can be constructed as $\mathbb{F}_3[t]/\langle t^2 - 2 \rangle$. Hence, any element of the field has the form
$a_0 + a_1t + \langle t^2 - 2\rangle$ with $a_i \in \mathbb{F}_3$. Hence, there are $3^2 = 9$ elements.
\\\\
In a similar manner, we know that $\mathbb{F}_2(\alpha)$ can be constructed as $\mathbb{F}_2[t]/\langle t^2+t+1 \rangle$. Hence any element of the field has the form
$a_0 + a_1t + \langle t^2+t+1\rangle$ with $a_i \in \mathbb{F}_2$. Hence there are $2^2 = 4$ elements.

\section*{Chapter 5. Degree}
\subsection*{Exercise 5.1.9}
Write out the addition and multiplication tables of $\mathbb{F}_2(\alpha)$.
\\\\
\textbf{Solution:}
The tables are straightforward, using modulo arithmetic and the irreducible polynomial evaluated at $\alpha$.\\
    \begin{center}
        \renewcommand\arraystretch{1.3}
        \setlength\doublerulesep{0pt}
        \begin{tabular}{r||*{4}{2|}}
            + & 0 & 1 & \alpha & 1+\alpha \\
            \hline\hline
            0 & 0 & 1 & \alpha & 1+\alpha \\ 
            \hline
            1 & 1 & 0 & 1+\alpha & \alpha \\ 
            \hline
            $\alpha$ & \alpha & 1+\alpha & 0 & 1 \\ 
            \hline
            $1+\alpha$ & 1+\alpha & \alpha & 1 & 0 \\ 
            \hline
        \end{tabular}
    \end{center}
    \vspace{0.5 cm}
    \begin{center}
        \renewcommand\arraystretch{1.3}
        \setlength\doublerulesep{0pt}
        \begin{tabular}{r||*{4}{2|}}
            $\times$ & 0 & 1 & \alpha & 1+\alpha \\
            \hline\hline
            0 & 0 & 0 & 0 & 0 \\ 
            \hline
            1 & 0 & 1 & \alpha & 1+\alpha \\ 
            \hline
            $\alpha$ & 0 & \alpha & 1+\alpha & 1\\ 
            \hline
            $1+\alpha$ & 0 & 1+\alpha & 1 & \alpha \\ 
            \hline
        \end{tabular}
    \end{center}

\subsection*{Exercise 5.1.13}
Give an example of to show that the inequalityin Corollary 5.1.12 can be strict. Your example can be as trivial as you like.
\\\\
\textbf{Solution:}
We choose our fields and hence extensions to be $\mathbb{C}:\mathbb{R}:\mathbb{Q}$. We also choose $\beta = \sqrt2 \in \mathbb{C}$.
The minimal polynomial of $\sqrt2$ over $\mathbb{Q}$ is $m = t^2 - 2$, 
then deg$_\mathbb{Q}(\beta) = [\mathbb{Q}(\beta):\mathbb{Q}] = 2$.\\\\
Similarly, the minimal polynomial of $\sqrt2$ over $\mathbb{R}$ is $m = t - \sqrt2$, 
then deg$_\mathbb{R}(\beta) = [\mathbb{R}(\beta):\mathbb{R}] = 1$.\\\\
Hence we have that $[\mathbb{R}(\beta):\mathbb{R}] < [\mathbb{Q}(\beta):\mathbb{Q}] \quad \square$

\subsection*{Exercise 5.1.16}
Let $M:K$ be a field extension and $\alpha$ a transcendental element of $M$. Can every element of $K(\alpha)$ be represented
as a polynomial in $\alpha$ over $K$?
\\\\
\textbf{Solution:}
We have that $K(\alpha) = \left\{ \frac {f(\alpha)}{g(\alpha)} : f,g \in F[t]\right\}$, which is just $K(t)$, the field
rational expressions. Therefore it is not polynomial is $\alpha$ over $K$. 

\subsection*{Exercise 5.1.20}
Show that a field extension whose degree is a prime number must be simple.
\\\\
\textbf{Solution:}
Let $M:K(\alpha):K$ be field extensions where $M$ and $K$ are arbitrary fields, $\alpha \in M$, and $[M:K] = p$, where $p$ is prime. By \textbf{Theorem 5.1.17 (iii)} we have $[M:K] = [M:K(\alpha)][K(\alpha):K]$.
Hence, we must have that $[K(\alpha):K] = 1$ or $p$,
however, we also know that $K(\alpha) \neq K$, hence $[K(\alpha):K] = p$, and therefore,
$[M:K(\alpha)]=1$, which by \textbf{Example 5.1.3} tells us $M=K(\alpha)$.
Hence $M:K$ is a simple.

\subsection*{Exercise 5.1.23}
Generalize Example 5.1.22. In other words, what
general result does the argument of Example 5.1.22 prove, not involving
the particular numbers chosen there?
\\\\
\textbf{Solution:}
Let $M:K$ be a field extension and $\alpha_1,...,\alpha_n \in M$. If $gcd(deg_K(\alpha_1)\\,...,deg_K(\alpha_n))=1$ (i.e., coprime),
then we have that, $[K(\alpha_1,...,\alpha_n):K] = [K(\alpha_1):K]...[K(\alpha_n):K]$ 

\subsection*{Exercise 5.2.5}
Let $M:K$ be a field extension and $K \subseteq L \subseteq M$. In the proof of Proposition 5.2.4, I said that if $L$ is a
subfield of $M$ then $L$ is a $K$-linear subspace of $M$. Why is that true? And is the converse also true? Give proof or a counterexample.  
\\\\
\textbf{Solution:}
We know that $M$ acts as a vector space over $K$. If $L$ is a subfield of $M$, then we can similarly conclude that $L$ acts as a vector space over $K$.
Since we have that $L$ is a subset of $M$ (a subfield) we can conclude that $L$ is a linear ($K$-linear) subspace of $M$ (by definition of a linear subspace).
\\\\ 
The converse is not true.

\subsection*{Exercise 5.2.8}
Let $M:K$ be a field extension and write $L$ for the set of elements of $M$ algebraic over $K$. By imitating the proof of Proposition 5.2.7,
prove that $L$ is a subfield of $M$.
\\\\
\textbf{Solution:}
We have that $L = \{\alpha \in M: [K(\alpha):K]<\infty\}$.\\
Then $\forall \alpha,\beta \in L,\ [K(\alpha,\beta):K] \leq [K(\alpha):K][K(\beta):K] < \infty$\\
Now $\alpha+\beta \in K(\alpha,\beta)$, so $K(\alpha+\beta) \subseteq K(\alpha,\beta)$, hence \\
$[K(\alpha+\beta):K] \leq [K(\alpha,\beta):K]<\infty$, giving $\alpha+\beta \in L$. Similarly, $\alpha \cdot \beta \in L$.\\
Then $\forall \alpha \in L,\ [K(-\alpha):K] = [K(\alpha):K]<\infty$, giving $-\alpha \in L$.
Similarly, $1/\alpha \in L$ (if $\alpha \neq 0$), and clearly $0,1 \in L \quad \square$

\subsection*{Exercise 5.3.7}
Find an example of Lemma 5.3.6 where $[LL':L]=2$, and another where $[LL':L]=1$.
\\\\
\textbf{Solution:}
If we let $L = \mathbb{Q}(\sqrt2)$ and $L' = \mathbb{Q}(\sqrt3)$, we then get $LL' = \mathbb{Q}(\sqrt2,\sqrt3)$.
Then $[\mathbb{Q}(\sqrt2,\sqrt3):\mathbb{Q}(\sqrt3)] = 2$.\\\\
If we let $L = \mathbb{Q}(\sqrt4)$ and $L' = \mathbb{Q}(\sqrt3)$, we then get $LL' = \mathbb{Q}(\sqrt4,\sqrt3)$.
Then $[\mathbb{Q}(\sqrt4,\sqrt3):\mathbb{Q}(\sqrt3)] = 1$.


\section*{Chapter 6. Splitting fields}
\subsection*{Exercise 6.1.5}
Show that if a ring homomorphism $\psi$ is injective then
so is $\psi_*$, and if $\psi$ is an isomorphism then so is $\psi_*$.
\\\\
\textbf{Solution:}
We have that $\psi: R \rightarrow S$ and $\psi_*:R[t] \rightarrow S[t]$. Since $\psi$ is injective we that
$\forall x,y \in R$ if $\psi(x)=\psi(y) \implies x=y$. Then we choose $f, f' \in R[t]$ and assume that
$\psi_*f = \psi_*f'$. From \textbf{Definition 3.1.7} we then have that:
\begin{equation*}
    \begin{aligned}
        \psi_*f &= \psi_*f'\\
        \psi_* \left(\sum_{i} a_it^i \right) &= \psi_* \left(\sum_{i} b_it^i \right)\\
        \sum_{i} \psi(a_i)t^i &= \sum_{i} \psi(b_i)t^i\\
        \psi(a_i) &= \psi(b_i)\\
        \implies a_i &= b_i 
    \end{aligned}
\end{equation*}
Hence we have that $f=\sum_{i}a_it^i = f'$. Hence $\psi_*$ is injective.
\\\\
If $\psi$ is an isomorphism then $\psi$ is both surjective and injective. We have just shown that
$\psi_*$ is injective, so we show that it is also surjective to prove it is an isomorphism. We know that
$\phi_*$ is surjective $\iff \forall s \in S[t] \ \exists r \in R[t]:\psi_*r=s$. We choose $s \in S[t]$ and
let $e$ be the identity homomorphism. Then we have:
\begin{equation*}
    \begin{aligned}
        s&=es\\
        &=(\psi_*\psi_*^{-1})s\\
        &=\psi_*(\psi_*^{-1}s)\\
        &=\psi_*\left(\psi_*^{-1}\left(\sum_{i}a_it^i\right)\right)\\
        &=\psi_*\left(\sum_{i}\psi^{-1}(a_i)t^i\right)\\
        &=\psi_*r
    \end{aligned}
\end{equation*}
where $\psi^{-1}(a_i) \in R$ exists since $\psi$ is an isomorphism, and $r=\sum_{i}\psi^{-1}(a_i)t^i \in R[t]$.
Hence $\psi_*$ is also surjective, hence it is an isomorphism.

\subsection*{Exercise 6.2.7}
Show that (ii) can equivalently be replaced by: `if $L$ is a subfield of $M$
containing $K$, and $f$ splits in $L$, then $L=M$'.
\\\\
\textbf{Solution:}
We first show ($\implies$). We have that $M= K(\alpha_1,...,\alpha_n)$ and so $M:K$ is well defined.
We then take a basis $\alpha_1,...,\alpha_n$ of $M$ over $K$. Then we have that every subfield $L$ of $M$
containing $K$ is a $K$-linear subspace of $M$. So if $\alpha_1,...,\alpha_n \in L$, which would mean that $f$ splits in $L$,
then $L=M$.
\\\\
We then show ($\Longleftarrow$). We have that $K \subseteq L = M$, and $f(t)=\beta(t-\alpha_1)\cdot\cdot\cdot(t-\alpha_n)$
for some $n \ge 0$ and $\beta, \alpha_1,...,\alpha_n \in L=M$. Then the result follows trivially from \textbf{Proposition 5.2.4}. 

\subsection*{Exercise 6.2.9}
In Example 6.2.8(iii), I said that $\mathbb{Q}(\xi, \omega\xi, \omega^2\xi) = \mathbb{Q}(\xi,\omega)$.
Why is that true?
\\\\
\textbf{Solution:}
$\omega=e^{2\pi i /3} = \frac{-1+i\sqrt{3}}{2}$, and hence $\omega^2 = \frac{-1-i\sqrt{3}}{2}$. and hence we see that $\omega^2$ is a rational multiple
of $\omega$, hence $\mathbb{Q}(\xi, \omega\xi, \omega^2\xi) = \mathbb{Q}(\xi,\omega)$.
\\\\
As an aside, we know that $\mathbb{Q}(\xi)= \left\{a + b\xi + c\xi^2: a,b,c \in \mathbb{Q} \right\}$, and hence $\{1, \xi, \xi^2\}$
forms a basis for $\mathbb{Q}(\xi):\mathbb{Q}$. Similarly, $\{1, \omega\}$ forms a basis for $\mathbb{Q}(\xi, \omega): \mathbb{Q}(\xi)$.
By \textbf{Theorem 5.1.17 (Tower Law)(i)} we then have that $\{1, \xi, \xi^2, \omega, \xi\omega, \xi^2\omega\}$ forms a basis for $\mathbb{Q}(\xi, \omega):\mathbb{Q}$.

\subsection*{Exercise 6.2.12}
Why does the proof of \textbf{Proposition 6.2.11} not show that there are \emph{exactly} $[M:K]$ isomorphisms $\phi$ extending $\psi$?
How could you strengthen the hypothesis in order to obtain that conclusion?
\\\\
\textbf{Solution:}
It can be strengthened by ...

\subsection*{Exercise 6.3.2}
Check that this really does define a group.
\\\\
\textbf{Solution:}
Firstly we have that $Gal(M:K) = Aut(M:K)$.
By definition we have $Aut(M:K)= \{f: M:K \rightarrow M:K \ |\ f \ \text{is an isomorphism of } M:K\}$
and $\circ: M:K \times M:K \rightarrow M:K$.
We show that the pair $(Aut(M:K), \circ)$ is a group.\\\\
Firstly, for $f,g \in Aut(M:K)$ and $\forall a,b \in M:K$, we have that:
\begin{equation*}
    \begin{aligned}
        (g \circ f) (ab) & = g(f(ab))\\
        & = g(f(a)f(b))\\
        & = g(f(a))g(f(b))\\
        & = (g \circ f)(a)(g \circ f)(b)
    \end{aligned}
\end{equation*}
Since $f,g$ are bijective by defintion, $g \circ f$ is bijective, and by above is a homomorphism, hence it is an automporhism.
\\\\
We now show associativity. $\forall f,g,h \in Aut(M:K)$ and $a \in M:K$ we have:
\begin{equation*}
    \begin{aligned}
        ((h \circ g) \circ f) (a) & = h(g(f(a)))\\
        & = h(g \circ f(a))\\
        & = h \circ (g \circ f(a))\\
        & = (h \circ (g \circ f))(a) \quad \square
    \end{aligned}
\end{equation*}
\\\\
Next, we check for an identity. $\forall f \in Aut(M:K)$ and $e_{M:K}: (M:K) \rightarrow (M:K) \ : a \mapsto a$ we have, 
$f \circ e_{M:K} = e_{M:K} \circ f  = f$. Hence $e_{M:K}$ is the identity element.
\\\\
Finally, we check for the inverse. $\forall f \in Aut(M:K)$, we have that $f \circ f^{-1} = e_{M:K} = f^{-1} \circ f$.
This follows by definition since $f$ is an isomorphism.

\subsection*{Exercise 6.3.4}
Prove that $Gal(\mathbb{Q}(e^{2\pi i /3}):\mathbb{Q}) = \{ id, \kappa\}$, where $\kappa(z)=\bar z$.
\\\\
\textbf{Solution:}
We know that the identity is an automorphism of $\mathbb{Q}(e^{2\pi i /3})$ over $\mathbb{Q}$. 
By \textbf{Lemma 1.1.2}, since $\mathbb{Q} \subset \mathbb{R}$ we have that $\kappa$ is also an automorphism of $\mathbb{Q}(e^{2\pi i /3})$ over $\mathbb{Q}$.
\\\\
Hence we have that $\{ id, \kappa\} \subseteq Gal(\mathbb{Q}(e^{2\pi i /3}):\mathbb{Q})$.
We also know that $\mathbb{Q}(e^{2\pi i /3}):\mathbb{Q} = \mathbb{Q}(\frac{-1+i\sqrt{3}}{2}):\mathbb{Q}=  \mathbb{Q}(i\sqrt{3}):\mathbb{Q}$
\\\\
We let $\theta \in Gal(\mathbb{Q}(e^{2\pi i /3}):\mathbb{Q})$. Since $\theta$ is a homomorphism we have that:
\begin{equation*}
    \begin{aligned}
        (\theta(i\sqrt{3}))^2 &= \theta((i\sqrt{3})^2)\\
        &= \theta(-3)\\
        &= -\theta(3)\\
        &= -3
    \end{aligned}
\end{equation*}
Hence $\theta(i \sqrt{3}) = \pm i\sqrt{3}$. If $\theta(i\sqrt{3}) = i\sqrt{3}$ then $\theta = id$,
and if $\theta(i\sqrt{3}) = -i\sqrt{3}$ then $\theta=\kappa$. Hence $Gal(\mathbb{Q}(e^{2\pi i /3}):\mathbb{Q}) = \{id, \kappa\}$.

\subsection*{Exercise 6.3.11}
I skipped two small bits in that proof: `$\theta$ is surjective because $\sigma$ is a permutation' (why?), and
`You can check that $\theta$ is a homomorphism of fields'. Fill in the gaps.
\\\\
\textbf{Solution:}
This follows by definition of permutation. A permutation is a bijective map from a set to itself, hence it is surjective.
$\theta$ is a homomorphism of fields.
\\\\
Secondly, by \textbf{Definition 6.3.5} $Gal_K(f)$ is $Gal(SF_K(f):K)$. Then by \textbf{Definition 6.3.1} we know that
an element of $Gal(M:K)$ is an isomorphism $\theta:M\rightarrow M$, hence by defintion we have that
$\theta \in Gal_K(f)$ is a homomorphism of fields.

\section*{Chapter 7. Preparation for the fundamental theorem}
\subsection*{Exercise 7.1.4}
What happens if you drop the word ‘irreducible’ from
Lemma 7.1.2? Is it still true?
\\\\
\textbf{Solution:}

\subsection*{Exercise 7.1.4}
What happens if you drop the word ‘irreducible’ from
Lemma 7.1.2? Is it still true?
\\\\
\textbf{Solution:}

\subsection*{Exercise 7.2.1}
Try to find an example of an irreducible polynomial
of degree $d$ with fewer than $d$ distinct roots in its splitting field.
\\\\
\textbf{Solution:}
An irreducible polynomial over a field of characteristic 0 has distinct roots in its splitting field.
Therefore we must consider field of characteristic $p >0$, where $p$ is prime. Hence if we have the field extension $\mathbb{F}_p(t):\mathbb{F}_p(t^p)$,
and we consdier $t \in \mathbb{F}_p(t)$ its minimal polynomial over $\mathbb{F}_p(t^p)$ is $X^p -t^p = (X-t)^p$. We get to the last step from the Frobenius automorphism.

\subsection*{Exercise 7.2.8}
Check one or two of the properties in Lemma 7.2.7.
\\\\
\textbf{Solution:}
We check the additive property.\\\\
We let $f(t) = \sum_{i=0}^{n} a_it^i \in K[t]$ and $ g(t) = \sum_{i=0}^{n} b_it^i \in K[t]$.
Then we have that $f(t)+g(t) = \sum_{i=0}^{n} (a_i+b_i)t^i = \sum_{i=0}^{n} c_it^i \in K[t]$, where
$c_i = (a_i+b_i)$. Then by \textbf{Definition 7.2.6} we have that $D(f+g)(t) = \sum_{i=1}^{n} ic_it^{i-1} = \sum_{i=1}^{n}i(a_i+b_i)t^{i-1} = \sum_{i=1}^{n} ia_it^{i-1} + \sum_{i=1}^{n} ib_it^{i-1} = Df + Dg \in K[t]$

\subsection*{Exercise 7.2.15}
Let $M:L:K$ be field extensions. Show that if $M:K$ is algebraic then so are $M:L$ and $L:K$.
\\\\
\textbf{Solution:}
By definition of an algebraic extension, we have that if $M:K$ is algebraic then $\forall \alpha \in M \ \exists f \neq 0 \in K[t] : f(\alpha)=0$. Since we have that $M$ is a field extension of $L$
it must contain all of $L$, therefore we have that $\forall \alpha \in L \ \exists f \neq 0 \in K[t]: f(\alpha)=0$, hence $L:K$ is algebraic. Similarly, since $L$ extends $K$
any $f \neq 0 \in K[t]$ must exist in $L[t]$, hence we that $\forall \alpha \in M \ \exists f \neq 0 \in L[t]: f(\alpha)=0$, hence $M:L$ is algebraic.

\subsection*{Exercise 7.3.2}
Using Lemma 7.3.1, show that every automorphism of a field is an automorphism over its prime subfield. In other words,
$Aut(M) = Gal(M:K)$ whenever $M$ is a field with prime subfield $K$.
\\\\
\textbf{Solution:}
By \textbf{Lemma 7.3.1} we have that $\forall S \subseteq Aut(M), \ Fix(S) \subseteq M$. Since $K$ is the prime subfield of $M$ we have that $K \subseteq Fix(S) \subseteq M$, and we also have that $K \subseteq Fix(S) \subseteq M:K$. Hence we have that:
\begin{equation*}
    \begin{aligned}
        Aut(M) &= \{ S: Fix(S) \subseteq M\}\\
        &= \{S: K \subseteq Fix(S) \subseteq M \}\\
        &= \{S: K \subseteq Fix(S) \subseteq M:K\}\\
        &= \{S: Fix(S) \subseteq M:K\}\\
        &= Gal (M:K) \quad \square
    \end{aligned}
\end{equation*}

\subsection*{Exercise 7.3.5}
Find another example of Theorem 7.3.3.
\\\\
\textbf{Solution:}
We follow \textbf{Example 7.3.4}. If we have $\kappa: \mathbb{Q}(\sqrt2) \rightarrow \mathbb{Q}(\sqrt2)$ representing complex conjugation,
then $H=\{id, \kappa\}$ is a subgroup of $Aut(\mathbb{Q}(\sqrt2))$. By \textbf{Theorem 7.3.3}, we have that $[\mathbb{Q}(\sqrt2): \text{Fix}(H)] \le |H| = 2$.
Since $\text{Fix}(H) = \mathbb{Q}$, and we know that $[\mathbb{Q}(\sqrt2):\mathbb{Q}]=2$, the inequality holds.


\section*{Chapter 8. The fundamental theorem of Galois theory}
\subsection*{Exercise 8.1.4}
Prove the first half of Lemma 8.1.2(i).
\\\\
\textbf{Solution:}
We assume that $L_1 \subseteq L_2$ and let $\phi \in Gal(M:L_2)$. Then we have that $\phi(\alpha) = \alpha \ \forall \alpha \in L_2$.
Hence we have that $\phi(\alpha) = \alpha \ \forall \alpha \in L_1$, hence $\phi \in Gal(M:L_1)$. Therefore $Gal(M:L_1) \supseteq Gal(M:L_2)$.

\subsection*{Exercise 8.1.5}
Draw a diagram like Figure 8.1 for the second half of
Lemma 8.1.2(i).
\\\\
\textbf{Solution:}
\begin{equation*}
\begin{tikzcd}
    & 1 \arrow[d, dash] & M \arrow[d, dash]\\
    & H_1 \arrow[d, dash] & Fix(H_1) \arrow[d, dash]\\
    & H_2 \arrow[d, dash] & Fix(H_2) \arrow[d, dash]\\
    & Gal(M:K) & K
\end{tikzcd}
\end{equation*}

\subsection*{Exercise 8.1.5}
Draw a diagram like Figure 8.1 for the second half of
Lemma 8.1.2(i).
\\\\
\textbf{Solution:}
We have that $M=SF_K(t^p-u)$ and $K=\mathbb{F}_p(u)$, hence as in \textbf{Example 7.2.19 (ii)} we have that $[M:K] = deg_K(\alpha) = p$, where $p$ is prime.
This means that $|\mathfrak{F}| = p$, and hence by the tower law we know that there are no trivial intermediate fields, hence $\mathfrak{F}=\{M,K\}$.
We have that $|Gal(M:K)| = Gal_K(t^p-u)$ and by by \textbf{Corollary 6.3.14} we that $Gal_K(t^p-u)\ | \ k! \leadsto  Gal_K(t^p-u) \ | \ 1!$, hence $|Gal(M:K)|=1$.
This means that $\mathfrak{G} = \{Gal(M:K)\}=\{id\}$. Therefore we can celarly see that it is impossible for there to be mutually inverse fucntion between
$\mathfrak{F}$ and $\mathfrak{G}$.

\subsection*{Exercise 8.1.5}
In this particular example, one can also see more
directly that all the extensions in (8.3) are normal. How?
\\\\
\textbf{Solution:}
By \textbf{Theorem 7.1.5} we that $SF_K(f) \iff M:K$ is finite and normal, where $M = SF_K(f)$. For each of the extensions shown in the diagram,
we can choose an $f$ e.g., $(t^2-2, t^2+1, t^2+2)$, thereby making each extension a splitting field. Therefore, we get by the theorem above,
that each extension is finite and normal. Even simpler, each of the extensions is of degree 2, hence they are normal extensions.

\subsection*{Exercise 8.3.3}
Show that every such $H$ must contain $p^2$.
\\\\
\textbf{Solution:}

\subsection*{Exercise 8.3.4}
I took a small liberty in the sentence beginning
‘The same argument’, because it included an inequality but the
previous argument didn’t. Prove the statement made in that sentence.
\\\\
\textbf{Solution:}
The statement is considering $\phi$ of order 2 only.
If $[\mathbb{Q}(\xi,i):\mathbb{Q}(\alpha)] = 2$, then we are in agreement with the previous argument.
The only other case is then that $[\mathbb{Q}(\xi,i):\mathbb{Q}(\alpha)] = 1$, in which case $\mathbb{Q}(\alpha) = \mathbb{Q}(\xi, i)$,
and $\phi = id$ which is of order 1, and hence, $Fix\langle \phi \rangle = \mathbb{Q}(\xi, i)$, by the fundamental theorem.

\subsection*{Exercise 8.3.5}
Choose one of $\mathbb{Q}(\xi^2), \mathbb{Q}(i) or \mathbb{Q}(\xi^2i)$ and do the same as I did for it as I just did for $\mathbb{Q}(\xi^2,i)$.
\\\\
\textbf{Solution:}
We choose $L=\mathbb{Q}(\xi^2)$. 
This gives us \[ G / \langle k, \rho^2 \rangle  \cong \text{Gal}(\mathbb{Q}(\xi^2):\mathbb{Q}) \]
The left hand side is the quotient of $D_4$ by a subgroup isomorphic to $C_2 \times C_2$. As we can observe, it has order 2.
Hence $G/ \langle k, p^2 \rangle \cong C_2$. On the other hand, $\mathbb{Q}(\xi^2)$ is the splitting field over $\mathbb{Q}$ of
$t^2 -2$. This is due to the fact the $\xi^2 = (\sqrt[4]{2})^2 = \sqrt2$. We know trivially that $\text{Gal}_{\mathbb{Q}}(t^2-2) = \text{Gal}(\mathbb{R}: \mathbb{Q}) \cong C_2$.
This confirms the isomorphism.

\section*{Chapter 9. Radicals}
\subsection*{Exercise 9.1.3}
Firstly, we know that the prime subfield of $\mathbb{C}$ is $\mathbb{Q}$, and as such every subfield of $\mathbb{C}$ cointains $\mathbb{Q}$.
Hence if we have $\mathbb{Q}^{rad}_a$ and $\mathbb{Q}^{rad}_b$ fullfilling (9.2) we can always find a $\mathbb{Q}^{rad}_c = \mathbb{Q}^{rad}_a \cap \mathbb{Q}^{rad}_b$
satisfying (9.2) since we can always have that $\mathbb{Q}^{rad}_c = \mathbb{Q}$, which satisfies (9.2).
\\\\
\textbf{Solution:}

\subsection*{Exercise 9.1.7}
In the last sentence of that proof, how exactly does it `follow'?
\\\\
\textbf{Solution:}
We have that $\phi , \theta \in Gal_{\mathbb{Q}}(t^n - 1) = Gal(SF_{\mathbb{Q}}(t^n - 1): \mathbb{Q}) = Gal(\mathbb{Q}(\omega): \mathbb{Q})$.
Then by \textbf{Lemma 4.3.6} we have that if $(\theta \circ \phi)(a)$ = $(\phi \circ \theta)(a)$, then $\theta \circ \phi = \phi \circ \theta$.

\subsection*{Exercise 9.1.10}
What does the proof of Lemma 9.1.8 tell you about the eigenvectors and eigenvalues of the elements of $Gal_K(t^n-a)$?
\\\\
\textbf{Solution:}

\subsection*{Exercise 9.1.11}
Use Lemmas 9.1.6 and 9.1.8 to show that $Gal_{\mathbb{Q}}(t^n - a)$ is solvable for all $a \in \mathbb{Q}$.
\\\\
\textbf{Solution:}
A group is solvable if it can be constructed from abelian groups through extensions.
Hence we have $\{1\} = G_0 \unlhd G_1 \unlhd ... \unlhd G_k = G$ such that $G_{i-1}$ is normal in $G_i$ and $G_{i} / G_{i-1}$
is an abelian group, for $i = 1, 2, ..., k$.
\\\\
We have that $Gal_{\mathbb{Q}}(t^n-a) = Gal(SF_{\mathbb{Q}}(t^n - a): \mathbb{Q})$.
\\\\
If $a=0$ then $Gal_{\mathbb{Q}}(t^n) = Gal(SF_{\mathbb{Q}}(t^n): \mathbb{Q}) = Gal(\mathbb{Q}: \mathbb{Q}) = \{1\}$ which is trivially solvable. 
\\\\
For $a \neq 0$, we need to show that $Gal_{\mathbb{Q}}(t^n-a)$ is abelian. If we choose a positive rational root $\xi$ of $t^n - a$ and let $\omega = e^{2\pi i /n}$, then
all the roots of $t^n - a$ are $\xi, \omega\xi, ..., \omega^{n-1}\xi$. Hence $SF_{\mathbb{Q}}(t^n - a)$ contains $(\omega^i\xi)/\xi \ \forall i$, therefore $t^n-1$ splits in $SF_{\mathbb{Q}}(t^n - a)$.
Hence we have $\mathbb{Q} \subseteq SF_{\mathbb{Q}}(t^n-1) \subseteq SF_{\mathbb{Q}}(t^n-a)$.
\\\\
By \textbf{Lemma 9.1.6} we have that $Gal_{\mathbb{Q}}(t^n-1)$ is abelian.  Let $K = SF_{\mathbb{Q}}(t^n-1)$, then by \textbf{Lemma 9.1.8} $Gal_K(t^n-a) = Gal(SF_K(t^n-a):K)$ is abelian.
Since all the extensions are splitting fields, we also know they are all normal.
Hence we have $\{1\} \unlhd Gal(K:\mathbb{Q}) \unlhd Gal(SF_{K}(t^n-a): K)$.
\\\\
We know that $Gal(K:\mathbb{Q})/\{ 1 \} = Gal(K:\mathbb{Q})$ which is abelian.
We also know that $Gal(SF_{K}(t^n-a):K)/Gal(K:\mathbb{Q})$ is abelian since the quotient group of an abelian group is also abelian.
Therefore $Gal_{\mathbb{Q}}(t^n - a)$  is solvable.

\end{document}
